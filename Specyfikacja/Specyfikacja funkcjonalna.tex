\documentclass[]{article}
\usepackage{ifxetex,ifluatex}
\ifnum 0\ifxetex 1\fi\ifluatex 1\fi=0 % if pdftex
\usepackage[T1]{fontenc}
\usepackage[utf8]{inputenc}
%\setcounter{secnumdepth}{0}
\usepackage[table,xcdraw]{xcolor}
\usepackage[margin=1.5in]{geometry}
\usepackage[tableposition=top]{caption}
\usepackage{tabularx}
\usepackage{hyperref}
\hypersetup{
colorlinks=true,
linkcolor=blue,
filecolor=magenta,
urlcolor=cyan,
}



\title{Specyfikacja funkcjonalna projektu zespołowego \textbf{AiSD GR2}}
\author{Martyna Jakubowska, Hubert Nakielski, Artur Prasuła}
\date{Grudzień 2020}


\begin{document}
    \maketitle


    \section{Opis ogólny}

    \subsection{Nazwa programu} % Hubert
    Patients Transit Visualizer

    \subsection{Poruszany problem} % Martyna
    Program, na podstawie dostarczonych danych, ma za zadanie pomóc zespołowi ratowników z Karetek Pogotowia (KP) przewozić pacjentów (P) do Ośrodków Medycznych (OM). Program dla każdego wprowadzonego pacjenta będzie wskazywał najbliższy w danej chwili pacjentowi Ośrodek Medyczny i symulował przetransportowanie go do wspomnianego obiektu do póki nie znajdzie wolnego miejsca lub nie odwiedzi wszystkich OM.

    \subsection{Użytkownik docelowy}
    Program dedykowany jest służbie zdrowia do planowania najbardziej optymalnych tras karetek.
    Użytkownikami programu będą dyspozytorzy, którzy wydają rozkazy co do tras karetek.


    \section{Opis funkcjonalności}

    \subsection{Jak korzystać z programu} % Martyna

    \subsection{Możliwości programu} % Artur


    \section{Format danych i struktura plików}

    \subsection{Struktura katalogów} % Hubert
    \textcolor{orange}{Proj\_zesp\_AiSD\_2020Z\_GR1}

    |---\textcolor{orange}{src}

    |\hspace{4mm} |-----\textcolor{orange}{Transit\_visualizer}

    |\hspace{15mm}|-----\textcolor{orange}{Models} - \textit{przechowuje wszystkie obiekty związane z logiką programu}

    |\hspace{15mm}|-----\textcolor{orange}{Views} -\textit{ odpowiada za wizualizację mapy i poszczególnych przewozów}

    |\hspace{15mm}|-----\textcolor{orange}{Controllers} -\textit{ zajmuje się obsługą żądań użytkownika}

    |\hspace{15mm}|-----\textcolor{orange}{Content} -\textit{ zawiera potrzebne do wizualizacji obrazki}

    |

    |

    |---\textcolor{orange}{test}

    \hspace{5mm} |-----\textcolor{orange}{Transit\_visualizer}

    \hspace{16mm}|-----\textcolor{orange}{Models}

    \hspace{16mm}|-----\textcolor{orange}{Controllers}

    \subsection{Przechowywanie danych w programie} % Hubert
    Dane wejściowe są przechowywane w listach (ArrayList) zawierających poszczególne obiekty. \\
    Każdy Ośrodek Medyczny (OM) reprezentowany jest przez obiekt typu\textit{ Hospital}, a wszystkie obiekty tego typu gromadzone są w liście. Dodatkowo OM zawiera listę dróg \textit{(Distance}) do innych, niekoniecznie wszystkich, OM.\\
    Obiekty (tj. pomniki, muzea, \ldots) przechowywane są w liście zawierającej obiekty typu \textit{Facility}.\\
    Pacjenci, reprezentowani przez obiekt \textit{Patient}, również przechowywani są w liście.\\

    Klasy obiektów i reprezentowane w nich dane:\\


    \textit{Hospital}
    \begin{itemize}
        \item  \textcolor{gray}{int} id
        \item  \textcolor{gray}{String} nazwa
        \item  \textcolor{gray}{int} współczynnik x
        \item \textcolor{gray}{int}  współczynnik y
        \item \textcolor{gray}{int}  liczba wszystkich łóżek
        \item \textcolor{gray}{int}  liczba dostępnych łóżek
    \end{itemize}

    \textit{Facility}
    \begin{itemize}
        \item  \textcolor{gray}{int} id
        \item  \textcolor{gray}{String}  nazwa
        \item  \textcolor{gray}{int}  współczynnik x
        \item \textcolor{gray}{int}  współczynnik y
    \end{itemize}

    \textit{Distance}
    \begin{itemize}
        \item \textcolor{gray}{int}  id
        \item  \textcolor{gray}{int} id pierwszego OM |
        \item  \textcolor{gray}{int} id drugiego OM |
        \item \textcolor{gray}{double}  odległość (jednostka reprezentująca czas)
    \end{itemize}

    \textit{Patient}
    \begin{itemize}
        \item  \textcolor{gray}{int} id
        \item \textcolor{gray}{int}  współczynnik x
        \item \textcolor{gray}{int}  współczynnik y
    \end{itemize}

    \subsection{Dane wejściowe} % Artur

    \subsection{Dane wyjściowe} % Martyna
    Program pokazuje wynik działania w graficznym interfejsie użytkownika w postaci przewiezionych pacjentów do wskazanych przez program OM lub komunikatu o brakujących wolnych miejscach w OM.


    \section{Scenariusz działania programu}

    \subsection{Scenariusz ogólny} % Hubert

    \subsection{Scenariusz szczegółowy} % Wszyscy

    \subsection{Ekrany działania programu} % Martyna


    \section{Testowanie}
    Poszczególne klasy programu zostaną przetestowane za pomocą testów jednostkowych. Do tego posłuży nam narzędzie JUnit. Współpraca klas w programie oraz działanie graficznego interfejsu użytkownika zostanie przetestowane przez nas ręcznie w trakcie implementacji.

\end{document}
